\documentclass[12pt]{article} % Default font size is 12pt, it can be changed here

\usepackage[spanish]{babel} % Configura el idioma del documento a español

\usepackage[a4paper, margin=1in]{geometry}

\usepackage{graphicx} % Required for including pictures

\usepackage{url}

\usepackage{float} % Allows putting an [H] in \begin{figure} to specify the exact location of the figure
\usepackage{wrapfig} % Allows in-line images such as the example fish picture

\linespread{1.2} % Line spacing

%\setlength\parindent{0pt} % Uncomment to remove all indentation from paragraphs

\graphicspath{{./imagenes/}} % Specifies the directory where pictures are stored

%para escribir en español
\usepackage[utf8x]{inputenc}

\usepackage{amsmath}
\usepackage{multirow}
\usepackage{longtable} % Para tablas que pueden ocupar más de una página
\usepackage{array} % Para opciones avanzadas de tablas
\usepackage{booktabs} % Para mejorar la calidad visual de las tablas
\usepackage{listings}
\usepackage{fancyvrb}
\DefineVerbatimEnvironment{Verbatim}{Verbatim}
{breaklines=true} % Activa el ajuste automático de líneas
\usepackage{caption}

%opening

\begin{document}
	\begin{titlepage}
		\newcommand{\HRule}{\rule{\linewidth}{0.5mm}} % Defines a new command for the horizontal lines, change thickness here
		
		\center % Center everything on the page
		
		\textsc{\LARGE Programación Lógica}\\[1.5cm] % Name of your university/college
		\textsc{\Large Curso 2025}\\[0.5cm] % Major heading such as course name
		\textsc{\large Grupo 01}\\[0.5cm] % Minor heading such as course title
		
		\HRule \\[0.4cm]
		{\huge \bfseries Informe - Laboratorio 2}\\[0.4cm] % Title of your document
		\HRule \\[1cm]
		
		\begin{minipage}{0.4\textwidth}
			\begin{flushleft} \large
				\emph{Autores:}\\
				\textsc{Gonzalo Apkarian Alonso \\ 5.494.604-2}\\ % Your name
				\textsc{Diego Pisa Sánchez \\ 5.510.635-2}\\ % Your name
				\textsc{Leonardo Pesce López \\ 5.471.535-2}\\
				\textsc{Mauricio Morales Gonzalez \\ 5.278.642-4}
			\end{flushleft}
		\end{minipage}
		~
		\begin{minipage}{0.4\textwidth}
			\begin{flushright} \large
				\emph{Profesores:} \\ 
				Luis \textsc{Chiruzzo} \\ 
				Aiala \textsc{Rosá} \\ 
				Juan Pablo \textsc{Conde}
			\end{flushright}
		\end{minipage}\\[3cm]
		
		% Logo
		\begin{figure}[H]
			\centering
			\includegraphics[width=0.2\textwidth]{logo.png} % Ajusta el tamaño aquí (0.2\textwidth es un ejemplo)
			\label{fig:logo}
		\end{figure}
		
		% Fecha
		{\large \today}\\[1cm] % Puedes cambiar \today por una fecha específica si lo prefieres
		
		\vfill % Esto asegura que el contenido se distribuya adecuadamente en la página
		
	\end{titlepage}
	
	\section{Módulos}
	
	Para este laboratorio no se usaron módulos extra, se implementaron todos los predicados en el mismo archivo principal. Esta decisión se tomó debido a que se tornó una complicación decidir como modularizar de manera eficiente el programa y no se contaba con mucho tiempo
	
\end{document}